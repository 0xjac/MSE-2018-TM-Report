\chapter{Strategies Evaluation}
\label{chap:strategies}

\section{Modelisation}
\subsection{Metrics}
A way to rationaly compare strategies is needed in order to discuss their
relative strengths and weaknesses. Three main characteristics will be used for
that:
\begin{itemize}
        \item latency;
        \item number of nodes;
        \item overhead.
\end{itemize}

\subsubsection{Latency}
The latency is the number of messages needed to finalize a block. Ideally, you
want to have a latency as low as possible to reach finality as soon as possible.
The latency is a way to measure liveness in a blockchain system. If it is low,
then the system is considered more "live", as less messages are needed in order
to confirm a transaction, and therefore less time.
\todo{schema for latency}

\subsubsection{Number of nodes}
The number of nodes is quite straightforward; it is the number of nodes that can
be included in the validator set. This number should be as high as possible to
guarantee decentralization and therefore safety.
\todo{schema for number of nodes}

\subsubsection{Overhead}
The overhead is the number of messages that are sent over the network between
one step of the consensus and the next. It should be as low as possible to keep
the costs in bandwidth low.
\todo{schema for overhead}


\subsection{Tradeoff triangle/Trilemma}

\subsection{Model}

\section{Strategies}
The following strategies were proposed in order to visit the entierty of the
trade-off triangle:
\begin{itemize}
        \item randomness;
        \item round-robin;
        \item double round-robin;
        \item overhead.
\end{itemize}
These strategies should allow one to visit the whole triangle and to discuss
their respective strength and weaknesses. The following sections describe the
strategies as well as their expected locations in the triangle.

\subsection{Round-robin}
The first strategy that comes to mind is a simple round-robin. Nodes send
messages one after the other, in a fixed order.

\subsection{Randomness}
The next strategy is the simplest to think of: complete randomness. Using fixed
probability density functions, nodes chose when to create messages and to which
other validator to send them.

\subsection{Double Round-robin}
In this setting, two nodes send messages at the same time, in a fixed order. If
the two nodes that send messages at the same step are at opposite places in the
set of validators \todo{explain better}, the latency to finality is supposedly
half as much as the simple round-robin strategy. The overhead is however
doubled.

\subsection{Maximal Overhead}
This strategy is the most expensive in terms of bandwidth; at each step, each
validator sends a message to the others. This example strategy should give a
baseline value for the maximum overhead that is reachable in the tradeoff
triangle.

\section{Experimentations}
Over the duration of this thesis, the \texttt{core-cbc} library has included a
test framework called \textit{proptest}. The testing framework that has been
implemented includes ways to simulate the behavior of the Casper protocol over
multiple nodes and thousands \todo{numbers} of blocks. At the time of the
writing, the simulations do not include networking latencies.

\todo{schema with what to measure}
\todo{how the measurements take place in the code}

\section{Visualization}

\todo{talk about sampling}
\section{Analysis}
