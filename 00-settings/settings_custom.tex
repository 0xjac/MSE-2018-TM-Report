% ======================== = Custom Settings ========================

\setlength{\parindent}{15pt} \setlength{\parskip}{0.0pt plus 1.0pt}


% \providecommand*{\listingautorefname}{Listing}
% \renewcommand{\lstlistingname}{Code}% Listing -> Algorithm
\def\lstlistingautorefname{Alg.}

% Create a new environment for breaking code listings across pages.
\newenvironment{longlisting}{\captionsetup{type=listing}}{}

\usepackage{float} \usepackage{pdfpages} \usepackage{emptypage}
\usepackage{amsfonts} \usepackage{dirtytalk} \usepackage{mathtools}
\usepackage{nccmath} \usepackage{tabularx} \usepackage[table]{xcolor}
\usepackage{hhline} \usepackage{listings} \newtheorem{theorem}{Theorem}[section]
\newtheorem{definition}{Definition}[section] \newtheorem{lemma}[theorem]{Lemma}
\newtheorem{corollary}[theorem]{Corollary}
\newtheorem{postulate}[theorem]{Postulate}

%% \todo{} command.
%
% Outputs red TODOs in the document. Requires \usepackage{color}.
%
% Usage: \todo{Document the TODO command.}
%
% Comment out second line to disable.
\newcommand{\todo}[1]{} \renewcommand{\todo}[1]{{\color{red} TODO: {#1}}}

%\newcommand{\eq}[1]{Eqn.\hspace{0.7mm}\ref{#1}}
\newcommand{\sect}[1]{Sec.\hspace{0.7mm}\ref{#1}}
\newcommand{\ssec}[1]{Subsec.\hspace{0.7mm}\ref{#1}}
\newcommand{\fig}[1]{Fig.\hspace{0.7mm}\ref{#1}}
\newcommand{\tab}[1]{Table\hspace{0.7mm}\ref{#1}}

\newcommand{\doublefigure}[4]{%
    \begin{figure}[h]
        \begin{center}
        \begin{array}{cc}

        \includegraphics[width=0.5\linewidth]{#1} &
        \includegraphics[width=0.5\linewidth]{#2}

        \end{center}
        \end{array}
        \caption{#3}
        \captionsetup{justification=centering}
        \label{#4}
    \end{figure}
}

\newcommand{\triplefigure}[5]{%
    \begin{figure}[h]
        \begin{center}
        \begin{array}{cc}

        \includegraphics[width=0.5\linewidth]{#1} &
        \includegraphics[width=0.5\linewidth]{#2} \\

        \includegraphics[width=0.5\linewidth]{#3} & \\

        \end{center}
        \end{array}
        \caption{#4}
        \captionsetup{justification=centering}
        \label{#5}
    \end{figure}
}

%\newcommand{\doublefigure}[4]{%
%    \begin{figure}[h]
%        \centering
%
%        \minipage{0.49\columnwidth}
%        \includegraphics[width=\linewidth]{#1}
%        %\caption{Histogram of the number of nodes. Round-robin and all receivers strategies}
%        \endminipage\hfill
%        
%        \minipage{0.49\columnwidth}
%        \includegraphics[width=\linewidth]{#2}
%        %\caption{Histogram of the number of nodes. Round-robin and all receivers strategies}
%        \endminipage\hfill
%
%        \caption{#3}
%        \captionsetup{justification=centering}
%        \label{#4}
%    \end{figure}
%}

%\newcommand{\triplefigure}[5]{%
%    \begin{figure}[h]
%        \centering
%
%        \minipage{0.5\columnwidth}
%        \includegraphics[width=\linewidth]{#1}
%        %\caption{Histogram of the number of nodes. Round-robin and all receivers strategies}
%        \endminipage\hfill
%
%        \minipage{0.5\columnwidth}
%        \includegraphics[width=\linewidth]{#2}
%        %\caption{Histogram of the number of nodes. Round-robin and all receivers strategies}
%        \endminipage\hfill
%
%        \minipage{0.5\columnwidth}
%        \includegraphics[width=\linewidth]{#3}
%        %\caption{Histogram of the number of nodes. Round-robin and all receivers strategies}
%        \endminipage\hfill
%
%        \caption{#4}
%        \captionsetup{justification=centering}
%        \label{#5}
%    \end{figure}
%}
